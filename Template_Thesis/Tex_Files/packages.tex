\usepackage[utf8]{inputenc}         % Permite escribir códigos especiales

% AJUSTES DEL IDIOMA
\usepackage[spanish]{babel}
\decimalpoint
\renewcommand{\spanishtablename}{Tabla}
\addto\extrasenglish{
    \renewcommand{\subsubsectionautorefname}{Section}
    \renewcommand{\subsectionautorefname}{Section}
    \renewcommand{\sectionautorefname}{Section}
}
\usepackage[title]{appendix}
\renewcommand{\appendixname}{Anexos}
\renewcommand{\appendixtocname}{Anexos}
\renewcommand{\appendixpagename}{Anexos}


% GEOMETRIA
\usepackage[paper=A4]{typearea}
\usepackage[includemp,
            top=2 cm,
            left = 1.2 cm, 
            right = 1.2 cm,
            bottom=2 cm,
            headsep = 3.5 cm,
            marginparwidth=2 cm,
            marginparsep=0.4 cm]{geometry}


% INDENTACION
\usepackage{indentfirst}
\setlength{\parskip}{5pt}       
\setlength{\parindent}{0pt}


% ESPACIADO
\usepackage{setspace}
\spacing{1.2}
\let\ph\mlplaceholder % shorter macro


% GENERAR PDF/A y otras cosas del pdf
\usepackage[a-1b]{pdfx}
\usepackage[pdftex]{graphicx}
    \graphicspath{
      {./Figures/Portada_HF/}
      {./Figures/01/}
      {./Figures/02/}
      {./Figures/03/}
      {./Figures/04/}
      {./Figures/05/}
      {./Figures/06/}
      {./Figures/A_01/}
      {./Figures/A_02/}
    }
\usepackage{pdfpages}               % Incluir PDF diferente tamaño


% URLS Y LINKS
\hypersetup{hidelinks}    % Hide borders links
\usepackage{url}
\Urlmuskip=0mu plus 1mu


% EXPRESIONES MATEMATICAS Y SÍMBOLOS
\usepackage{textcomp}               % Symbols in text
\usepackage{amsmath}
\usepackage{amssymb}
\usepackage{mathtools}
\usepackage[T1]{fontenc}
\usepackage{mathtools}
\usepackage{mathrsfs}
\usepackage{derivative}
\usepackage{amsmath,amssymb}
\usepackage{float}
\DeclareMathOperator{\Tr}{Tr}
\usepackage{bigints}
\usepackage{gensymb}    % Para hacer los circulitos de grados
\usepackage{eurosym}
\usepackage{fontawesome}


% INCLUIR CÓDIGO EN EL DOCUMENTO
\usepackage{listings}
\renewcommand{\lstlistingname}{Listado}		% Para que las listas sean es español
\usepackage[framed,numbered,autolinebreaks,useliterate]{mcode}      % MATLAB CODE
% \lstset{
%     inputencoding=utf8,
%     literate={Á}{{\'A}}1 {á}{{\'a}}1 {É}{{\'E}}1 {é}{{\'e}}1 {Í}{{\'I}}1 {í}{{\'i}}1 {Ó}{{\'O}}1 {ó}{{\'o}}1 {Ú}{{\'U}}1 {ú}{{\'u}}1 
%     }

%\lstset{
%  basicstyle         = \mlttfamily,
%  escapechar         = ",
%}
%\usepackage[numbered]{matlab-prettifier}


% ACRONIMOS
% https://tex.stackexchange.com/questions/25520/how-can-i-use-the-latex-acronym-package-and-optionally-create-an-acronym-list-i
% https://sourceforge.net/p/texstudio/discussion/907839/thread/7ced058c/  -> SI NO APARECEN LOS ACRONIMOS
\usepackage[acronym,smallcaps]{glossaries}		% ,nonumberlist
% \loadglsentries[\acronymtype]{./Tex_Files/acronyms.tex}
\makeglossaries
%\usepackage{nomencl}
%\makenomenclature
%\usepackage{etoolbox}
%\renewcommand\nomgroup[1]{%
%  \item[\bfseries
%  \ifstrequal{#1}{P}{Propiedades físicas}{%
%  \ifstrequal{#1}{S}{Señales ópticas}{%
%  \ifstrequal{#1}{F}{Fibra óptica}{%
%  \ifstrequal{#1}{C}{Material compuesto}{}}}}%
% ]}


% BIBLIOGRAFÍA
%\usepackage{biblatex}
%\addbibresource{references.bib}
%\setlength\parindent{0pt}


% DISEÑO DE TABLAS Y FIGURAS 
% \usepackage{subfigure}
% \usepackage{subfloat}
\usepackage{caption}
\usepackage{subcaption}		% https://tex.stackexchange.com/questions/295456/texstudio-beginsubfigure-unrecognized-command
\usepackage{svg}
\usepackage{import}
\usepackage{longtable}
\usepackage{multirow}
\usepackage{multicol}
\usepackage{threeparttable}
\usepackage{booktabs}
\usepackage{tabu}
\usepackage{bigstrut}
\usepackage{tabularx}
    \makeatletter
    \def\hlinewd#1{%
    \noalign{\ifnum0=`}\fi\hrule \@height #1 \futurelet
    \reserved@a\@xhline}
    \makeatother
\usepackage{placeins}  %para poder poner Floatbarrier


% ENUMERACIONES
\usepackage{enumerate}
\usepackage{enumitem}  % Selecionar la forma del item


% NOTAS PIE DE PAGINA
\usepackage[colorinlistoftodos]{todonotes}  % TO DO
\newcommand\blfootnote[1]{%
  \begingroup
  \renewcommand\thefootnote{}\footnote{#1}%
  \addtocounter{footnote}{-1}%
  \endgroup
}


% DEFINICIÓN DE COLORES 
\usepackage{xcolor}
\usepackage{colortbl}


% COMANDOS
\providecommand\phantomsection{}    % Para añadir phantomsections al indice


% LANDSCAPE
\usepackage{pdflscape}
\usepackage{everypage}
\usepackage{lipsum}
% Landscape configuration
\newcommand{\Lpagenumber}{\ifdim\textwidth=\linewidth\else\bgroup
  \dimendef\margin=0 %use \margin instead of \dimen0
  \ifodd\value{page}\margin=\oddsidemargin
  \else\margin=\evensidemargin
  \fi
  \raisebox{\dimexpr -\topmargin-\headheight-\headsep-0.5\linewidth}[0pt][0pt]{%
    \rlap{\hspace{\dimexpr \margin+\textheight+\footskip}}}%
\egroup\fi}
\AddEverypageHook{\Lpagenumber}%
% Code
%\begin{landscape}
% Text
%\end{landscape}


% MISCELANEO
\usepackage{cite}
\usepackage{csquotes} % Cita en el texto
\usepackage{comment} % Comentar en bloque
\usepackage{lastpage} % Citar la última página
\usepackage{relsize} % Tamaños relativos
\usepackage{bm} % Para poner negrita math tablas
\usepackage{printlen}
\usepackage{afterpage}	% añadir algo despues de una pagina